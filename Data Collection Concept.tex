\documentclass[options]{article}

 \usepackage[
    top    = 2.75cm,
    bottom = 2.50cm,
    left   = 4.00cm,
    right  = 3.50cm]{geometry}

\usepackage[parfill]{parskip}

\title{Mathematics! Should it still be a prerequisite to pursue computer science?}
\author{Rukundo Jonathan \thanks{supervisor: Ernest Mwebaze}}
\date{%
    Makerere University\\%
    May 17, 2017
}


\begin{document}
\begin{titlepage}
\maketitle
\end{titlepage}





\section{\textbf{ Introduction}} 
Computer science is often difficult to define. This is probably due to the unfortunate use of the word “computer” in the name. As you are perhaps aware, computer science is not simply the study of computers. Although computers play an important supporting role as a tool in the discipline, they are basically tools. 


\subsection{\textbf{Background}}
Computer science is the study of problems, problem-solving, and the solutions that come out of the problem-solving process. Given a problem, a computer scientist's goal is to develop an algorithm, a step-by-step list of instructions for solving any instance of the problem that might arise. Algorithms are solutions.\bigbreak

Briefly put, computer science can be thought of as the study of algorithms. While studying computer science, it becomes apparent that algorithms and mathematics go hand in hand. In the computer science curriculum, mathematics is almost omnipresent as students are introduced to various mathematical concepts such as calculus, numerical methods and discrete mathematics. \bigbreak

This explains why passing mathematics at the lower levels is a pre-requisite before one can study computer science in any reputable university across the globe. Therefore this gave birth to a belief that all computer science students should at least be fairly good at Mathematics. 


\subsection{\textbf{Problem Statement}}
This project will examine the impact of a computer science student's mathematical background on their academic performance and programming skills.


\subsection{\textbf{Objectives}}


\subsubsection{\textbf{Main Objective}} 
The main goal of this project is to determine whether one needs to be good at mathematics to excel in computer science. 


\subsubsection{\textbf{Specific Objectives}}

\begin{itemize}
  \item To collect all the data necessary to aid our research.
  \item To perform a thorough analysis on the collected data.
  \item To come up with a conclusion from the data analysis.
\end{itemize}


\subsection{\textbf{Scope}}
This research is aimed at computer science students at higher institutions of learning such as Makerere University.

\subsection{\textbf{Research Significance}}
This study is important because it aims at improving the computer science curriculum at higher institutions of learning.



\section{\textbf{Literature review}}

We looked at \cite{latexGuide}Makerere University's B.Sc. (Computer Science) program, and for a candidate to be admitted, he/she must have: At least a subsidiary pass in Mathematics in the Uganda Advanced Certificate of Education (UACE) or its equivalent and at least two principle passes at the same sitting in UACE in any of the following subjects: - Mathematics, Economics, Entrepreneur-ship, Geography, Physics, Chemistry, Biology, Agriculture, Technical Drawing.




\section{\textbf{Methodology}}
The proposed methodology consists of two phases, data collection and data analysis.\bigbreak
Data will be collected using ODK Collect, which will later on be uploaded to the ODK aggregate server to carry out all the required analysis. Different kinds of data (including images and GPS coordinates) will be collected, these include: 

\begin{itemize}
  \item Student’s name, recent photo and place of residence (including GPS coordinates)
  \item Employee status
  \item Programming level (beginner, intermediate and none)
  \item UACE mathematics result
  \item Current performance at the university (CGPA). 
\end{itemize}



\begin{thebibliography}{10} \bibitem{latexGuide} Makerere University, \emph{Computer science curriculum(May 2009)}, Available at \texttt{http://cit.mak.ac.ug/downloads/} \end{thebibliography}



\end{document}